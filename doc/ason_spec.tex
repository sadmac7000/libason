\documentclass[letterpaper]{article}
\usepackage[margin=1in]{geometry}
\usepackage{amsthm,amssymb,amsmath}
\title{ASON: A semantically complete superset of JSON\\\normalsize\vspace{5mm} Version 0.2}
\author{Casey Dahlin}
\begin{document}
\setlength{\parskip}{1em}
\setlength{\parindent}{0em}
\newtheorem{prule}{Rule}
\newcommand{\kvpair}[2]{``#1":\quad#2,\quad}
\newcommand{\kvpaire}[2]{``#1":\quad#2}

\maketitle

\section{Introduction}
JSON\footnote{RFC 4627 - http://www.ietf.org/rfc/rfc4627.txt?number=4627} has
become an increasingly popular data interchange format for a wide variety of
applications, from web services and RPC interfaces, to databases. Yet the JSON
spec has never specified a semantic for JSON values.

Meanwhile, NoSQL databases, some of which use JSON for data presentation or
more, are rising in popularity, yet there are no agreed upon semantics for
querying such a database.

ASON is a superset of JSON syntax, as well as a semantic. The acronym stands
for \textbf{A}lgebraic \textbf{S}erialized \textbf{O}bject \textbf{N}otation.
It is designed to represent not only JSON values, but families of values. It
includes a series of transforms which can allow a large, complex value or
composition of values, such as might represent an entire table in a database,
to be queried or made to conform to a schema.

It is hoped that ASON will serve not only these practical purposes, but also
open the door to a formalization for the transformation of unstructured data,
similar to what relational algebra has been for relational databases.

\section{Additional Syntax}
\begin{samepage}
We will assume a general familiarity with JSON syntax already. To JSON's
syntax, ASON adds three new invariant values, much like \(true\), \(false\),
and \(null\):

\begin{itemize}
	\item The Universe value (\(U\))
	\item The Wild value (\(*\))
	\item The Empty value (\(\emptyset\), also rendered as \(\_\))
\end{itemize}
\end{samepage}

These can appear anywhere that a JSON value could appear; as list items, values
in an object, or by themselves.

\subsection{Operators}
In addition to the new constants, ASON introduces four new operators; three
binary infix operators and one unary prefix operator.

Operators and parenthesis behave as structural characters do in the JSON
syntax\footnote{RFC 4267, section 2}.

\begin{samepage}
The operators, in order of descending precedence, are:

\begin{itemize}
\item Complement (\(!\)) A prefix operator.
\item Join (\(:\))
\item Intersect (\(\cup\) or \(\&\))
\item Union (\(\cap\) or \(|\))
\end{itemize}
\end{samepage}

\begin{samepage}
\subsection{Universal Object Syntax}
ASON introduces a subtype of object called a ``universal object." It is
represented identically to a normal object, except that after the list of
key-value pairs is one ``\(,\)" and one ``\(*\)". An example:
%
\begin{equation}
\{ \kvpair{num}{6} \kvpair{string}{``hello"} *\}
\end{equation}
\end{samepage}

Please note that we will typically use the term ``object" to refer to either
normal or universal objects.

\section{ASON Intuitively}
The purpose of this section is to give an intuitive grasp of ASON. The next
section will endeavor to rigorously formalize it. For this reason, this section
may gloss over or ignore several issues. Those implementing the specification
should have a thorough understanding of the formal transformations in ASON.

ASON relies on a bijection between ASON values and sets. All ASON values which
reflect typical singular JSON values map to sets of cardinality 1. We state
that two ASON values are equal \em if and only if \em their mapped sets are
equal.

\subsection{Operations in ASON}
We can think of ASON values as collections of JSON values, so the ASON value
\(\{\kvpaire{foo}{8}\}\) is a set containing only the identical JSON value.

The purpose of the \(\cup\) and \(\cap\) operators is thus apparent: they
combine and manipulate these singleton collections to form collections
containing multiple values. Their specific function is upon the underlying
sets, and analogous to the equivalent set theory operations: the union of two
ASON values is an ASON value whose underlying set is the union of each of the
values' underlying set. So the ASON value \(6\) is a collection containing
\(6\), and the ASON value \(6\cup7\) is a collection containing \(6\) and
\(7\).

\subsection{The Universe, Wild, and Empty Values}
The additional values we have added over JSON can all be explained in these
terms: \(\emptyset\) is a value underlied by the empty set, and represents a
collection containing nothing. \(U\) is the ASON universe, and is a collection
of all possible ASON values. \(*\) is similar to \(U\), except it does not
contain the value \(null\).

\begin{samepage}
\subsection{Universal Objects}
Universal objects are used to refer to collections of objects which all have
certain keys in common. For example, the collection of all objects which have a
key \(``foo"\) with value \(6\) would be:
%
\begin{equation}
\{\kvpair{foo}{6} *\}
\end{equation}
\end{samepage}

\subsection{Complementation}
The \(!\) operator also specifies a union of several values. Specifically it
specifies a union of all values not equal to the value it is applied to.

It is intuitive to say the collection of all values which are not \(6\) is
\(!6\). It is possibly more subtle to point out \(U = !\emptyset\), or \(* =
!null\), but these equalities should hold given our definition.

\subsection{Join}
Join operations can combine the keys of two objects, and also simulate the SQL
join operation when applied to unions of objects.

\section{Formalization}
\begin{samepage}
We will allow for numbers in JSON/ASON form to be transformed in ways that
mathematics implies yield identical results, for example \(6.0 = 6\) and \(+3 =
3\). We will also assume that characters in strings may be replaced with their
equivalent unicode escape sequences as specified by JSON, and vice versa.
Given this and the transformations on other values we will outline below, we
will define equality thus:
%
\begin{prule}
Two ASON values are equal \em if and only if \em they can each be transformed
into an identical representation.
\end{prule}
\end{samepage}

\begin{samepage}
Reflexively, this specification shall describe a series of transformations on
ASON values which do not alter their semantic value. We will also state that:
%
\begin{prule}
\label{sec:commute}
If a value \(A\) can be transformed into a value \(B\), then the value \(B\)
may be transformed into the value \(A\).
\end{prule}
\end{samepage}

And lastly:
%
\begin{prule}
\label{sec:assoc}
All operators are associative.
\end{prule}

\subsection{Basic Transforms}
These are intuitive rules understood by most existing JSON users and
implementations. We will formalize them here.

\begin{prule}
The order of two key-value pairs in an object may be exchanged.
\end{prule}

\begin{prule}
Values within objects or lists may be transformed.
\end{prule}

\subsection{Key Expansion}
\begin{prule}
\begin{equation}
\begin{split}
& \{\kvpaire{foo}{6} \}\\
= \quad& \{\kvpair{foo}{6} \kvpaire{bar}{null}\}
\end{split}
\end{equation}

A non-universal object may have a key-value pair with any key and value
\(null\) inserted in to it.
\end{prule}

\begin{prule}
\begin{equation}
\begin{split}
& \{\kvpair{foo}{6} *\}\\
= \quad& \{\kvpair{foo}{6} \kvpair{bar}{U} *\}
\end{split}
\end{equation}

A universal object may have a key-value pair with any key and value \(U\)
inserted in to it.
\end{prule}

Key expansion dramatically simplifies the definition of other operations.
Performing key expansion is often a crucial step in computing reduced ASON
values. Do however note that key expansion implies that there is no difference
between a key being set to \(null\) and a key not being present, at least
within non-universal objects. This is a logical, but not necessarily expected
semantic for JSON's \(null\) value.

Reversal of this transformation via rule \ref{sec:commute} is typically
referred to as \em key contraction \em .

\subsection{Obliteration}
\begin{prule}
\label{sec:oblit}
\begin{equation}
\begin{split}
{[}7,\quad8,\quad9,\quad\emptyset,\quad10] & = \emptyset \\
\{\kvpair{foo}{6}\kvpaire{bar}{\emptyset}\} & = \emptyset
\end{split}
\end{equation}
Any object or list containing a value which is equal to \(\emptyset\) may be
replaced with \(\emptyset\)
\end{prule}

Obliteration is somewhat counterintuitive, but is required to simplify other
operations. Without this transformation it might be provable that the
underlying set of a non-universal object has a cardinality greater than 1.
Consider our coming definition of intersection.

\subsection{Intersection Reduction}

\begin{prule}
\label{sec:interdist}
\begin{equation}
\begin{split}
& \Bigg( \{\kvpaire{foo}{6}\} \cup \{\kvpaire{bar}{7}\} \Bigg) \quad \cap \quad
\{\kvpaire{baz}{8}\} \\
= \quad& \{\kvpaire{foo}{6}\} \cap \{\kvpaire{baz}{8}\} \quad \cup \quad \{\kvpaire{bar}{7}\}
 \cap \{\kvpaire{baz}{8}\}
\end{split}
\end{equation}

The intersect operator may be distributed over the union operator.
\end{prule}

This isn't at all surprising, and if we've begun to intuit the analogy between
ASON union and intersection and set theory's implementations of the same, it
becomes obvious.

\begin{prule}
\label{sec:interelim}
\begin{equation}
\begin{array}{rclcl}
6 \quad&\cap&\quad 7 & = & \emptyset \\
\{\kvpaire{foo}{``bar"}\} \quad&\cap&\quad 7 & = & \emptyset \\
{[}4,\quad5,\quad6] \quad&\cap&\quad 7 & = & \emptyset \\
{[}4,\quad5,\quad6] \quad&\cap&\quad {[}7,\quad8] & = & \emptyset \\
{[}4,\quad5,\quad6] \quad&\cap&\quad \{\kvpaire{foo}{``bar"}\} & = & \emptyset \\
\end{array}
\end{equation}

The intersection of any two values which are not equal, and of which neither
can be expressed as a union of two or more non-equal, non-empty values, is
\(\emptyset\).
\end{prule}

This is a necessarily complex way of specifying what is ultimately a simple
concept: intersecting unequal values which are not collections of multiple
values yields an empty value. It becomes muddled when we attempt to express it
as purely transformation over syntax, as this specification aims to do.

\begin{prule}
\label{sec:interident}
\begin{equation}
\begin{array}{rclcl}
6 \quad&\cap&\quad 6 & = & 6 \\
\{\kvpaire{foo}{``bar"}\} \quad&\cap&\quad \{\kvpaire{foo}{``bar"}\}  & = & \{\kvpaire{foo}{``bar"}\} \\
{[}4,\quad5,\quad6] \quad&\cap&\quad {[}4,\quad5,\quad6] & = & {[}4,\quad5,\quad6] \\
\end{array}
\end{equation}

The intersection of any two values which are equal is equal to those values.
\end{prule}

The counterpart to the previous rule is much more cleanly specified. A value
intersected with itself is itself.

\begin{prule}
\label{sec:interlist}
\begin{equation}
{[}a,\quad b,\quad c] \quad\cap\quad {[}d,\quad e,\quad f]  =  {[}a\cap d,\quad
b\cap e,\quad c\cap f] \\
\end{equation}

The intersection of any two lists of the same length can be reduced to a list
which is also that length and where each element is the intersection of the
corresponding two elements in the operand lists.
\end{prule}

This can almost be inferred from the previous rules and some of the properties
of unions discussed later. In fact you can prove rules \ref{sec:interident} and
\ref{sec:interelim} for the specific case of lists of the same length with this
rule. Nonetheless this rule does clarify the details of intersection.

\begin{prule}
\label{sec:interobj}
\begin{equation}
\begin{array}{rlllll}
& \{\kvpair{foo}{a}&\kvpair{bar}{b}&\kvpaire{baz}{c}&&\}\quad\cap\\
& \{&\kvpair{bar}{d}&\kvpair{baz}{null}&\kvpaire{bam}{e}&\} \\
\\
=&\{\kvpair{foo}{a}&\kvpair{bar}{b}&\kvpair{baz}{c}&\kvpaire{bam}{null}&\}\quad\cap\\
 &\{\kvpair{foo}{null}&\kvpair{bar}{d}&\kvpair{baz}{null}&\kvpaire{bam}{e}&\} \\
\\
=&\Bigg\{\kvpair{foo}{a\cap null}&\kvpair{bar}{b\cap
d}&\kvpair{baz}{c\cap null}&\kvpaire{bam}{e\cap null}&\Bigg\}\\
\end{array}
\end{equation}

An intersection of two or more objects, universal or otherwise, can be replaced
with a single object, where that object is universal if and only if both of
the two operands were universal.
\begin{enumerate}
\item Use key expansion so both original objects have the same set of keys. The
result object will also have that set of keys
\item For each key, the result object will have the intersection of the two
values for that key in the operands.
\end{enumerate}
\end{prule}

This complements the previous rule, and rounds out our definition of
intersection. Again, it nicely parallels rule \ref{sec:interelim}, but does
resolve real ambiguity in the system.

\subsection{Join Reduction}

\begin{prule}
\begin{equation}
\begin{split}
& \Bigg( \{\kvpaire{foo}{6}\} \cup \{\kvpaire{bar}{7}\} \Bigg) \quad : \quad
\{\kvpaire{baz}{8}\} \\
= \quad& \{\kvpaire{foo}{6}\} : \{\kvpaire{baz}{8}\} \quad \cup \quad \{\kvpaire{bar}{7}\}
 : \{\kvpaire{baz}{8}\}
\end{split}
\end{equation}

The join operator may be distributed over the union operator.
\end{prule}

An issue of precedence, which should be well expected by now.

\begin{prule}
\begin{equation}
\begin{array}{rclcl}
6 \quad&:&\quad 7 & = & \emptyset \\
\{\kvpaire{foo}{``bar"}\} \quad&:&\quad 7 & = & \emptyset \\
{[}4,\quad5,\quad6] \quad&:&\quad 7 & = & \emptyset \\
{[}4,\quad5,\quad6] \quad&:&\quad {[}7,\quad8] & = & \emptyset \\
{[}4,\quad5,\quad6] \quad&:&\quad \{\kvpaire{foo}{``bar"}\} & = & \emptyset \\
\\
6 \quad&:&\quad null & = & 6 \\
\end{array}
\end{equation}

The joining of any two values which are not equal, and of which neither
can be expressed as a union of two or more non-equal, non-empty values, is
\(\emptyset\), unless one of those values is \(null\), in which case the join
may be expressed as the other value singly.
\end{prule}

We borrow this from the intersect operator, which join mirrors, and add an
exception, which wholly encapsulates how join differs from intersect.

\begin{prule}
\begin{equation}
\begin{array}{rclcl}
6 \quad&:&\quad 6 & = & 6 \\
\{\kvpaire{foo}{``bar"}\} \quad&:&\quad \{\kvpaire{foo}{``bar"}\}  & = & \{\kvpaire{foo}{``bar"}\} \\
{[}4,\quad5,\quad6] \quad&:&\quad {[}4,\quad5,\quad6] & = & {[}4,\quad5,\quad6] \\
\end{array}
\end{equation}

The join of any two values which are equal is equal to those values.
\end{prule}

Here intersect behaves identically.

\begin{prule}
\begin{equation}
{[}a,\quad b,\quad c] \quad:\quad {[}d,\quad e,\quad f]  =  {[}a: d,\quad
b: e,\quad c: f] \\
\end{equation}

The join of any two lists of the same length can be reduced to a list
which is also that length and where each element is the intersection of the
corresponding two elements in the operand lists.
\end{prule}

Again, similar to intersection.

\begin{prule}
\begin{equation}
\begin{array}{rlllll}
& \{\kvpair{foo}{a}&\kvpair{bar}{b}&\kvpaire{baz}{c}&&\}\quad:\\
& \{&\kvpair{bar}{d}&\kvpair{baz}{null}&\kvpaire{bam}{e}&\} \\
\\
=&\{\kvpair{foo}{a}&\kvpair{bar}{b}&\kvpair{baz}{c}&\kvpaire{bam}{null}&\}\quad:\\
 &\{\kvpair{foo}{null}&\kvpair{bar}{d}&\kvpair{baz}{null}&\kvpaire{bam}{e}&\} \\
\\
=&\Bigg\{\kvpair{foo}{a: null}&\kvpair{bar}{b:
d}&\kvpair{baz}{c: null}&\kvpaire{bam}{e: null}&\Bigg\}\\
\end{array}
\end{equation}

A join of two or more objects, universal or otherwise, can be replaced
with a single object, where that object is universal if and only if both of
the two operands were universal.
\begin{enumerate}
\item Use key expansion so both original objects have the same set of keys. The
result object will also have that set of keys
\item For each key, the result object will have the join of the two
values for that key in the operands.
\end{enumerate}
\end{prule}

We finish here the same way we finish in intersection.

\subsection{Union Reduction}
\begin{prule}
\begin{equation}
a\cup\emptyset = a
\end{equation}

Any value unioned with \(\emptyset\) can be reduced to that value singly.
\end{prule}

This more or less defines \(\emptyset\). Combined with rule \ref{sec:oblit}, it means
that \(\emptyset\) should almost always be able to be removed from an ASON
value by transformation.

\begin{prule}
\begin{equation}
a\cup a = a
\end{equation}

Any value unioned with itself can be reduced to that value singly.
\end{prule}

Interestingly, this parallels intersection.

\begin{prule}
\label{sec:objuniondist}
\begin{equation}
\begin{array}{rcl}

{[}a,\quad b,\quad c] \cup {[}a,\quad b,\quad d] &=& {[}a,\quad b,\quad c \cup d] \\
\\
\{\kvpair{foo}{a}\kvpaire{bar}{b}\} \cup \{\kvpair{foo}{a}\kvpaire{bar}{c}\}
&=& \{\kvpair{foo}{a}\kvpaire{bar}{b \cup c}\}

\end{array}
\end{equation}
Any union of lists or objects where the operands  differ in only one value may
be reduced to a version of that list or object where the differing value is a
union of the values within the two operands.
\end{prule}

This rule is key, but it is more often used in reverse. Unions can be
propagated upward such that there is only one top-level sequence of unions by
repeatedly exchanging values containing unions for unions of values.

\subsection{Complementation}
\begin{prule}
\label{sec:compelim}
\begin{equation}
\begin{array}{rcl}
	!!a &=& a
\end{array}
\end{equation}
Complementation of a complemented value is that value.
\end{prule}

\begin{prule}
\begin{equation}
\begin{array}{rcl}
	!a \cap a &=& \emptyset
\end{array}
\end{equation}
Intersecting a value with its complement yields \(\emptyset\).
\end{prule}

\begin{prule}
\label{sec:demorgan}
\begin{equation}
\begin{array}{rcl}
	! (a \cup b) &=& !a \cap !b
\end{array}
\end{equation}
A complement of a union of two values is equal to the intersection of the
complement of those two values separately.
\end{prule}

Complementation is defined by a series of simple rules: it is its own inverse,
complemented values don't intersect with the original values, and complementing
a union is equivalent to intersecting the complements of its operands (an
analogue to DeMorgan's law).

\subsection{Constants}
\begin{equation}
\begin{split}
U = &!\emptyset \\
* = &!null
\end{split}
\end{equation}

With complementation defined, the two constant values are easily expressed. In
fact they are revealed as syntactic sugar.

\section{Orders}

One of the most important reasons to transform an ASON value is to place it in
the simplest possible form. We divide ASON values into a series of numbered
\textit{orders}, which roughly identify how complex the simplest form of the
value is.

Each order is a superset of all the orders below it, so an order 0 value is
also an order 1 value. However, when we refer to ``the order of value B" we
typically refer to the minimally-applicable order. For the purposes of this
section, we will denote variables with their implied order as a subscript, so
variables referring to order 0 values will appear as \(a_0\) and soforth.

\subsection{Closure}

An important property of each order is \textit{closure}. An order is
\textit{closed} under a given operator if applying that operator to two values
of that order yields a value also of that order. For example, if \(a_n \cap
b_n\) is always a value of order \(n\) then order \(n\) is closed under
intersection.

In addition to the four ASON operators, we will also discuss whether values are
closed under \textit{containment}, that is, does using a value of a given order
as an element of a list or object also yield a value of that order. If so, that
order is closed under containment.

\subsubsection{Closure and Join}

Because the join operator is equivalent to the intersect operator in most
cases, closure under intersection should imply closure under join in all such
cases. The excepted case is joining with the \(none\) value, which will always
be closed, because \(a_n : none = a_n\), which is order \(n\). Therefore we can
ignore closure under join and simply discuss closure under intersection.

\subsection{Order 0}

Order 0 is the simplest order. We define it as follows:

\begin{itemize}
	\item All ASON values which are representable as valid JSON values are
		order 0.
	\item \(\emptyset\) is order 0.
\end{itemize}

\subsubsection{Closure under containment}

We take it as trivial that any JSON value can also be a value within another
containing JSON value, so closure under containment for JSON is trivial.

Rule \ref{sec:oblit} proves that containment of \(\emptyset\) must result in
\(\emptyset\), so containing \(\emptyset\) is also closed.

\subsubsection{Closure under intersection}

Rule \ref{sec:interelim} implies that intersecting two order 0 values will
either yield \(\emptyset\) if they are unequal, or their mutual value if they
are equal. The excepted case is if either can be expressed as a union of two
non-equal, non-empty values, which we take as trivially impossible for
JSON-equivalent values.

\subsection{Order 1}

Order 1 is any value expressable in the form:

\begin{equation}
a_0 \cup b_0
\end{equation}

Or the form:

\begin{equation}
a_1 \cup b_1
\end{equation}

\subsubsection{Closure under union}

Closure under union is implied in the definition of the order.

\subsubsection{Closure under containment}

Suppose one has an object containing a value that is order 1, which is to say
it is a union of multiple order 0 values. One can split that object by rule
\ref{sec:objuniondist} into multiple objects each of which has an order 0 value in
that position, and which are joined externally by a union. If the objects have
no other order 1 values then they are themselves order 0 and the union of them is
order 1. If they have other order 1 values we can apply the process recursively
until they no longer do and we are left with a chained union of several order 0
values.

\subsubsection{Closure under intersection}

Rule \ref{sec:interdist} states we can distribute an intersection across
unions. This means we can continue to distribute intersections until
we have a union of intersections of order 0 values. Order 0 is closed under
intersection so the result is simply a union of order 0 values, which is order
1.

\subsection{Order 2}
Order 2 contains order 1, as well as the complement of all order 1 values, and
the union of any other order 2 values.

\subsubsection{Closure under union}

Closure under union is implied in the definition of the order.

\subsubsection{Closure under complement}

Given an order 1 value \(a_1\), \(!a_1\) must be order 2, but \(!!a_1\) is
order 1, because the complementation can be eliminated given rule
\ref{sec:compelim}.

If an order 2 value is a union of other order 2 values, then by associativity
we can eventually express it as unions of a series of order 1 values, and a
series of complemented order 1 values. Example:

\begin{equation}
a_1 \cup b_1 \cup c_1 \cup ! d_1 \cup ! e_1 \cup ! f_1
\end{equation}

We can use rule \ref{sec:demorgan} and rule \ref{sec:assoc} to isolate the
inverted items and convert them to an intersection:

\begin{equation}
a_1 \cup b_1 \cup c_1 \cup ! ( d_1 \cap  e_1 \cap  f_1 )
\end{equation}

Order 1 is closed under union and intersection, so we can substitute:

\begin{equation}
g_1 \cup ! h_1
\end{equation}

Now let's complement this form:

\begin{equation}
	! (g_1 \cup ! h_1)
\end{equation}

We can apply rule \ref{sec:demorgan} again to get:

\begin{equation}
	! g_1 \cap h_1
\end{equation}

\(h_1\) is either an order 0 value or a union of one or more such values by
definition:

\begin{equation}
	! g_1 \cap ( i_0 \cup j_0 \cup k_0 )
\end{equation}

We can apply rule \ref{sec:interdist} to distribute:

\begin{equation}
	!g_1 \cap i_0 \cup !g_1 \cap j_0 \cup !g_1 \cap k_0
\end{equation}

Order 0 values can't be reflected by a union of one or more non-empty,
non-equal values, so the result of the above intersections must be just the
order 0 values by rule \ref{sec:interident} or \(\emptyset\) by
\ref{sec:interelim}. The result is a union of order 0 values, or an order 1
value.

\subsubsection{Closure under intersection}

Suppose the form:

\begin{equation}
a_2 \cap b_2
\end{equation}

We can rewrite this in the form:

\begin{equation}
! ( !a_2 \cup !b_2 )
\end{equation}

Order 2 is closed under complement and union, so the two variables are
rewritable as uncomplemented order 2 values:

\begin{equation}
! ( c_2 \cup d_2 )
\end{equation}

Order 2 is also closed under union:

\begin{equation}
!e_2
\end{equation}

And since, again, we are closed under complement, this is an order 2 value.

\subsubsection{Order of \(U\) and \(*\)}

Since \(null\) and \(\emptyset\) are order 0, and complements of order 0 values
are order 2, \(!null\) and \(!\emptyset\), also representable as \(U\) and
\(*\), are order 2.

\subsection{Order 3}
Order 3 contains all remaining values. This includes the universal object \({ *
}\), and values containing order 2 values.

\subsection{Order and ``Cardinality"}

One way to look at orders is to look at them in terms of ``cardinality." An
order 0 value has a cardinality of 1, which is to say it represents exactly one
JSON value.  Order 1 values can represent finite collections of JSON values,
Order 2 values can represent infinite collections of values, but their
complements always represent finite collections of values. Order 3 contains
values which represent infinite sets of values, but which may be complemented
to represent finite sets of values.

\end{document}
